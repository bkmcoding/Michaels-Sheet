\documentclass{article}
\usepackage{tikz}
\usetikzlibrary{shapes, arrows.meta, positioning}
\usepackage{geometry}

\geometry{a4paper, margin=1in}

\begin{document}

\section{Asymptotic Order Notation}

Let $f(x)$ and $g(x)$ be functions of a positive $x$.

\subsection{Big O Notation}
\textbf{Big O notation} describes an upper bound on the growth rate of a function.
\[ f(x) = O(g(x)) \]
when there is a positive constant $c$ such that,
\[ f(x) \le cg(x) \]
for all $x \ge n$.

\subsection{Big Omega Notation}
\textbf{Big Omega notation} describes a lower bound on the growth rate of a function.
\[ f(x) = \Omega(g(x)) \]
when there exist positive constants $c, n$ such that for all $x \ge n$ we have,
\[ f(x) \ge cg(x) \]

\subsection{Big Theta Notation}
\textbf{Big Theta notation} describes a tight bound on the growth rate of a function.
\[ \Omega(g(x)) = f(g(x)) = O(g(x)) \]
Then such a tight bound is stated as $f$ is big-$\Theta$ of $g$.
\[ f(x) = \Theta(g(x)) \]

\section{Question 1: Conditional Work Subroutine}

Consider the following Java subroutine:

\begin{verbatim}
public static void conditionalWork(int n) {
    for (int i = 0; i < n; i++) {
        if (Math.random() < 0.5) {
            taskA()
        } else {
            taskB()
        }
    }
}
\end{verbatim}

\subsection{Part 1}

If on a certain machine the function \textbf{taskA()} takes 2 seconds on each call in the loop and the function \textbf{taskB()} takes 4 seconds then for $n = 10$, what is the total number of expected seconds taken by the subroutine public static void conditionalWork(int n)?

The total number of expected seconds is 30 seconds.
\[ a(x) = \frac{x}{2} \]
\[ b(x) = 4(\frac{x}{2}) \]
% \[ result(10) = 2 \cdot 5 + 4 \cdot 5 = 30 \text{ seconds} \]

\subsection{Part 2}

The subroutine conditionalWork(int n) is ran on a much faster machine reducing the time taken by \textbf{taskA()} on each call down to $\frac{1}{2}$ of a second and \textbf{taskB()} now takes $\frac{1}{5}$ of a second. For $n = 10$, what is the new total number of seconds expected by the subroutine conditionalWork(int n)?

\[ A = \frac{1}{2} \]
\[ B = \frac{1}{5} \]
\[ result(10) = 25 \cdot \frac{1}{2} + 5 \cdot \frac{1}{5} = 5 \]
% \[ 0.2 + 1 = 1.2 \text{ seconds} \]

\end{document}